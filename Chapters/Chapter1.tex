% Chapter Template

\chapter{Research proposal}\label{chapter:firstchapter} % Main chapter title

\label{ChapterX} % Change X to a consecutive number; for referencing this chapter elsewhere, use \ref{ChapterX}

%----------------------------------------------------------------------------------------
%	SECTION 1
%----------------------------------------------------------------------------------------

\section{Introduction}\label{sec:firstsection}


% It is a good idea to have each sentence on a separate line, so that if you get feedback or changes from someone else
% the diffs will be much easier to manage
In this technological driven era, new technical advancement features in collaboration with human ability, many products are available that uses technology  still there exists a gap between technology and  areas where people are completely isolated to communicate with rest part of the world, interaction to get aware of what is going around in the world, is thus prime importance for vulnerable population. However, the trends are certainly going to change as we are coming up with an innovative and economically feasible concept of providing tsunami alert using a system of an advanced technology, so called ETC-Lali Low-Cost Communications and All-Hazard Early Warning System, it is thus claimed that this system can be the future of indo-pacific region.

\begin{figure}
\begin{centering}
\includegraphics[width=10cm,height=10cm,keepaspectratio]{Figures/dont-panic-e1534046233310.jpg}
\caption{The Hitch Hiker's Guide To The Galaxy (not to be confused with \cite{Reference1}. Image Credit David Strine (License: CC0)}
\label{fig:ThisFig}
\end{centering}
\end{figure}

%-----------------------------------
%	SUBSECTION 1
%-----------------------------------
\subsection{Project Focus  }

Natural disasters pose largest threat to the life of people living on earth. Certainly, the region that records highest number of natural disasters on earth, According to study of mental health asia, Indo pacific region specifically Asia is among the highest natural disaster prone area on planet earth, there are assorted reasons, out of them is the geographical location as eastern and south east countries lies   on pacific seismic belt  from Aleutian trench to Mariana trench, Likewise Eurasian plate is situated on Indian plates. So whenever a collision occurs in between these plates, there has been high destruction earthquakes causing damaging to human life and causing tsunami. Another, study of reference ind author endnote revealed the real world example of destruction caused by tsunami waves in Japan which killed above 15,000 people and costing the country economically around 235 billion dollars. A system for alert for tsunami is the current need. Although there exists many multi-hazard warning systems but the working, cost and maintenance of those systems can be covered by large organisation or government firms. For instance in India, Tsunami warning system was installed known as (IOTWMS) Indian Ocean Tsunami Warning & Mitigation System where twenty four centres for warnings in Indian oceans were setup.  The systems are working , However the risk of coastal hazards still cannot be successfully minimised as there were barriers in communication of warning which does not even reached to vulnerable population, other reasons were data collection and gathering of messages and lastly the lack of devices and equipment’s in the region.  Problem  - people do not have economical viable system for tsunami which can be  bought  as it cost a lot of money to  purchase install and maintain the system for people living coastal areas. So our project is aiming to provide the solution which can be economical for people to buy and install it.
The main focus of the project is to provide direct link to vulnerable population living in isolated islands and other area which are completely excluded from the rest part of world. The project aims to provide awareness to that part of area and will keep the locals update in their locality as ETC-Lali system will not only provide the alerts for tsunami, but at the same time providing the full entertainment by playing music and keeping them update by streaming daily news through radio programming.
 

%-----------------------------------
%	SUBSECTION 2
%-----------------------------------

\subsection{Subsection 2}
Morbi rutrum odio eget arcu adipiscing sodales.
Aenean et purus a est pulvinar pellentesque.
 Cras in elit neque, quis varius elit.
 Phasellus fringilla, nibh eu tempus venenatis, dolor elit posuere quam, quis adipiscing urna leo nec orci.
 Sed nec nulla auctor odio aliquet consequat.
 Ut nec nulla in ante ullamcorper aliquam at sed dolor.
 Phasellus fermentum magna in augue gravida cursus.
 Cras sed pretium lorem.
 Pellentesque eget ornare odio.
 Proin accumsan, massa viverra cursus pharetra, ipsum nisi lobortis velit, a malesuada dolor lorem eu neque.

%----------------------------------------------------------------------------------------
%	SECTION 2
%----------------------------------------------------------------------------------------

\section{Main Section 2}

Sed ullamcorper quam eu nisl interdum at interdum enim egestas.
 Aliquam placerat justo sed lectus lobortis ut porta nisl porttitor.
 Vestibulum mi dolor, lacinia molestie gravida at, tempus vitae ligula.
 Donec eget quam sapien, in viverra eros.
 Donec pellentesque justo a massa fringilla non vestibulum metus vestibulum.
 Vestibulum in orci quis felis tempor lacinia.
 Vivamus ornare ultrices facilisis.
 Ut hendrerit volutpat vulputate.
 Morbi condimentum venenatis augue, id porta ipsum vulputate in.
 Curabitur luctus tempus justo.
 Vestibulum risus lectus, adipiscing nec condimentum quis, condimentum nec nisl.
 Aliquam dictum sagittis velit sed iaculis.
 Morbi tristique augue sit amet nulla pulvinar id facilisis ligula mollis.
 Nam elit libero, tincidunt ut aliquam at, molestie in quam.
 Aenean rhoncus vehicula hendrerit.
